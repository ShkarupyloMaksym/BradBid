% !TEX TS-program = lualatex
% !TEX spellcheck = uk_UA
% !TEX options = --output-directory=build
% !TEX encoding = UTF-8

\documentclass[14pt]{extarticle}

\usepackage{fontspec}
\usepackage{polyglossia}
\setdefaultlanguage{ukrainian}
\usepackage{geometry}
\geometry{margin=1in}
\usepackage{hyperref}
\hypersetup{
  colorlinks=true,
  linkcolor=blue,
  urlcolor=blue
}

\begin{document}

\section{Покрокова інструкція з реалізації архітектури на AWS (SQS Edition)}

Нижче наведено детальний план реалізації MVP біржового сервісу на AWS у команді з двох осіб, адаптований під \textbf{Free Tier} (використання SQS замість Kinesis, відмова від NAT Gateway). У місцях, де має бути код, подано позначку \texttt{(place code here later)}.

\subsection{Розподіл ролей у команді}

\textbf{Учасник A (Інфраструктура + Frontend):}
\begin{itemize}
    \item Створення базової інфраструктури AWS (VPC Public-only, SG).
    \item Налаштування CloudFront, Route~53.
    \item Конфігурація Cognito User Pool та App Client.
    \item Розробка та деплой SPA/PWA.
    \item Моніторинг та керування витратами.
\end{itemize}

\textbf{Учасник B (Backend + Data Pipeline):}
\begin{itemize}
    \item Налаштування SQS FIFO, DynamoDB.
    \item Створення Lambda-функцій (ingest, matcher).
    \item Розгортання ElastiCache Redis (або Docker на EC2 Free Tier).
    \item Реалізація аналітичного пайплайну (S3, Glue, Athena).
    \item Налаштування спостереження (CloudWatch).
\end{itemize}

\subsection{Крок 1. Базовий проєкт}

\begin{enumerate}
    \item Створити GitHub репозиторій зі структурою:
    \begin{itemize}
        \item \texttt{/infra} --- Terraform.
        \item \texttt{/backend} --- Lambda функції.
        \item \texttt{/frontend} --- SPA.
        \item \texttt{/docs} --- документація.
    \end{itemize}
    \item Створити IAM користувачів для учасників.
    \item Додати скелет Terraform: \texttt{(place code here later)}.
\end{enumerate}

\subsection{Крок 2. Мережа та базова інфраструктура (Учасник A)}

\begin{enumerate}
    \item Створити VPC:
    \begin{itemize}
        \item \textbf{Тільки Public Subnets} (для економії на NAT Gateway).
        \item Internet Gateway для доступу Lambda до зовнішніх API.
    \end{itemize}
    \item Налаштувати Security Groups:
    \begin{itemize}
        \item Redis SG: дозволити трафік лише від Lambda SG (порт 6379).
        \item Lambda SG: вихідний трафік 0.0.0.0/0.
    \end{itemize}
    \item Місце для IaC: \texttt{(place code here later)}.
\end{enumerate}

\subsection{Крок 3. Ідентифікація та доступ (Учасник A)}

\begin{enumerate}
    \item Створити Cognito User Pool.
    \item Створити App Client для SPA.
    \item Налаштувати IAM ролі:
    \begin{itemize}
        \item ролі виконання Lambda (політики SQS, DynamoDB, Logs),
        \item роль authorizer для API Gateway.
    \end{itemize}
    \item Місце для конфігурації: \texttt{(place code here later)}.
\end{enumerate}

\subsection{Крок 4. API Gateway та CloudFront (Учасник A)}

\begin{enumerate}
    \item Створити API Gateway HTTP endpoint для REST API.
    \item Створити WebSocket API для стрімінгу котирувань.
    \item Підключити Cognito Authorizer.
    \item Створити CloudFront для SPA.
    \item \textit{(Видалено WAF для економії бюджету)}.
    \item Місце для конфігів: \texttt{(place code here later)}.
\end{enumerate}

\subsection{Крок 5. Черги та сховище (Учасник B)}

\begin{enumerate}
    \item Створити \textbf{SQS FIFO} черги:
    \begin{itemize}
        \item \texttt{orders.fifo} (для вхідних ордерів),
        \item \texttt{trades.fifo} (для виконаних угод).
    \end{itemize}
    \item Увімкнути \texttt{ContentBasedDeduplication} у чергах.
    \item Створити DynamoDB таблиці:
    \begin{itemize}
        \item \texttt{Orders},
        \item \texttt{Trades}.
    \end{itemize}
    \item Створити S3 бакет для аналітики.
    \item Місце для інфраструктури: \texttt{(place code here later)}.
\end{enumerate}

\subsection{Крок 6. Redis (Кеш) (Учасник B)}

\begin{enumerate}
    \item Створити ElastiCache Redis (\texttt{cache.t3.micro}) або EC2 з Docker.
    \item \textbf{Важливо:} Налаштувати Security Groups так, щоб Lambda з Public Subnet мала доступ до Redis (використовувати Private IP або VPC peering, якщо Redis у Private).
    \item Місце для TF: \texttt{(place code here later)}.
\end{enumerate}

\subsection{Крок 7. Lambda-функції (Учасник B)}

\subsubsection*{Ingest Lambda}
\begin{itemize}
    \item Тригер: API Gateway (REST \texttt{POST /orders}).
    \item Валідація JSON.
    \item Відправка повідомлення у \texttt{SQS: orders.fifo}.
    \item Використання \texttt{MessageGroupId} = торгова пара (напр. \texttt{BTC\_USD}) для гарантії порядку.
    \item Місце для коду: \texttt{(place code here later)}.
\end{itemize}

\subsubsection*{Matcher Lambda}
\begin{itemize}
    \item Тригер: \texttt{SQS: orders.fifo}.
    \item \textbf{Важливо:} \texttt{BatchSize=1} або \texttt{10} (для MVP краще 1, щоб уникнути складнощів з частковими помилками).
    \item Логіка:
    \begin{enumerate}
        \item Отримати ордер з SQS.
        \item Завантажити OrderBook з Redis.
        \item Знайти зустрічний ордер (Match).
        \item Оновити Redis (через Lua скрипт для атомарності).
        \item Зберегти результат у DynamoDB.
        \item Відправити повідомлення у \texttt{SQS: trades.fifo}.
    \end{enumerate}
    \item Місце для коду: \texttt{(place code here later)}.
\end{itemize}

\subsection{Крок 8. WebSocket повідомлення (Учасники A+B)}

\begin{enumerate}
    \item Matcher Lambda (або окрема Lambda-слухач \texttt{trades.fifo}) викликає API Gateway Management API.
    \item Розсилка оновлень підключеним клієнтам (Broadcast ticker).
    \item Місце для коду push-сервісу: \texttt{(place code here later)}.
\end{enumerate}

\subsection{Крок 9. Аналітичний шар (Учасник B)}

\begin{enumerate}
    \item Оскільки Firehose не підтримує прямий вхід з SQS:
    \item Варіант 1 (Простий): Matcher Lambda пише JSON безпосередньо у Firehose \texttt{->} S3.
    \item Варіант 2 (Дешевий): EventBridge Pipe читає \texttt{trades.fifo} \texttt{->} записує в S3 (або через Lambda).
    \item Glue crawler + Athena для SQL запитів по історії торгів.
    \item Місце для конфігів: \texttt{(place code here later)}.
\end{enumerate}

\subsection{Крок 10. Спостереження (Учасники A+B)}

\begin{enumerate}
    \item CloudWatch Logs для всіх Lambda.
    \item Метрики SQS: \texttt{ApproximateNumberOfMessagesVisible} (глибина черги).
    \item X-Ray (опційно, якщо вписується у Free Tier).
    \item Місце для JSON дашбордів: \texttt{(place code here later)}.
\end{enumerate}

\subsection{Крок 11. CI/CD (спільно)}

\begin{enumerate}
    \item GitHub Actions для \texttt{terraform apply} та деплою Lambda.
    \item Додати \texttt{terraform destroy} workflow для економії коштів.
    \item Місце для YAML: \texttt{(place code here later)}.
\end{enumerate}

\subsection{Крок 12. Локальна розробка та тести}

\begin{itemize}
    \item Локальні unit-тести для логіки матчингу (без AWS).
    \item Використання \texttt{localstack} для емуляції SQS/DynamoDB локально.
    \item Місце для тестів: \texttt{(place code here later)}.
\end{itemize}

\subsection{Крок 13. Фінальна демонстрація}

\begin{itemize}
    \item Показати "живу" торгівлю: створення ордеру -> оновлення графіку.
    \item Показати логи в CloudWatch та дані в S3/Athena.
    \item Пояснити вибір архітектури (SQS vs Kinesis) з точки зору вартості.
\end{itemize}

\end{document}
